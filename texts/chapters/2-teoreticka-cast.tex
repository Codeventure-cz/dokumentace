\chapter{Teoretická část}

\section{Pojmy}

\subsection{WebSocket}

\subsection{Frontend}

\subsection{Backend}

\subsection{Kompilace}

\subsection{Kontejner}
A container is a standard unit of software that packages up code and all its dependencies so the application runs quickly and reliably from one computing environment to another. \cite{Kontejner}

\subsection{Framework}

\subsection{Komponenta}
Komponenta je menší část aplikace, která poskytuje určitou funkcionalitu aplikace.

\subsection{Herní engine}

\subsection{Cloud}

\subsection{VPS}

\subsection{API}

\subsection{Škálování}

\subsection{JSON}

\subsection{Transpilace}

\subsection{Transakce}

\subsection{Cache}

\section{Technologie}

\subsection{Unity}
Unity je herní engine určený hlavně pro nováčky, ale i pro pokročilé programátory. Jako programovací jazyk Unity využívá C\# vyvýjený firmou Microsoft. Tento programovací jazyk je jednoduchý pro nováčky a je jedním z nejzsnažších programovacích jazyků v současnosti na naučení. \cite{Csharp} Dají se v něm vytvářet 2D i 3D aplikace. Výsledná hra je spustitelná na mnoha platformách, od počítačů až po mobilní zařízení.

\subsection{JavaScript}
JavaScript je moderní programovací jazyk využívající se jak na frontendu, ale i na backendu. Jazyk je tzv. interpretovaný jazyk, což znamená, že kód je spustitelný rovnou ze zdrojového kódu a není potřeba jej kompilovat.

\subsection{TypeScript}
TypeScript je nadstavbou JavaScriptu, která poskytuje typování kódu. Díky tomu je ve výsledné aplikaci méně chyb spojených s typy, protože jsou kontrolovány během vývoje a sestavování aplikace. Ve výsledku je kód transpilován zpět do JavaScriptu.
TODO domyslet

\subsection{React.js}
React.js je JavaScriptový framework pro vývoj uživatelské rozhraní neboli UI. Tvoří se s ním dynamické webové aplikace nebo mobilní aplikace. Vývojářem tohoto frameworku je firma Meta (dříve známá jako firma Facebook). UI je děleno na menší komponenty, které mají svoji funkcionalitu a dají se mezi sebou jednoduše zaměňovat. Díky tomu není například potřeba načíst celou stránku při přechodu na jinou stránku, ale stačí pouze zaměnit komponenty které se na stránce změnily (např hlavička a patička zůstávají stále načteny, ale obsah stránky se změní).

\subsection{Next.js}
Next.js je JavaScriptový framework stavící na frameworku React.js. Je vytvářen společností Vercel, která pod stejným názvem provozuje i vlastní Cloud. Tento framework usnadňuje vývojáři práci se stránkami a přidává nezpočet výhod nad použitím samothného Reactu.js. Hlavním lákadlem je takzvaný "Server Side Rendering" což znamená, že počáteční kód který, je poslán klientovi, je nejdříve vykreslen na serveru, pak je odeslán v neživé podobě klientovi a až po té, co je stránka načtena tak se z něj stává dynamická stránka pomocí React.js. Mezi další výhody patří například automatická optimalizace velikosti obrázků podle toho, jak velké je klient potřebuje, aby se nemusely posílat zbytečně velké a neoptimalizované obrázky.

\subsection{Node.js}
Node.js je backendové prostředí běžící na JavaScriptovém kódu. Výhodou oproti starším programovacím jazykům které běží na serveru např. PHP je, že aplikace neustále běží na pozadí. To umožňuje o dost jednoduší práci s časem (např. načasování určitých událostí neboli Cronjobů) nebo práci v reálném čase pomocí WebSocketů.

\subsection{GraphQL}
GraphQL je je jazyk pro API rozhraní sloužící pro jednoduchou komunikací mezi aplikací a serverem. Typy dotazů se dělí na 3 skupiny: Query: získání dat ze serveru, Mutation: změna dat na serveru a Subscription: zasílání dat v reálném čase pomocí WebSocketů.

\subsection{MongoDB}
MongoDB je dokumentová databáze, která je používaná na velké objemy dat. Data jsou zde ukládána do dokumentů ve formátu JSON. Databáze je jednoduše škálovatelná mezi více zařízeními v tak zvaném "Replica Setu" (data jsou kopírovaná mezi všemi zařízením, aby bylo možné při výpadku některého zařízení pokračovat dál v chodu aplikace). Replica set umožňuje i další pokročilé funkce jako například "transakce".

\subsection{Redis}
Redis je databáze fungující na principu klíč a hodnota. Pod nějakým klíčovým slovem si můžeme uložit hodnotu a následně si můžeme tuto hodnotu zobrazit po zadání tohoto klíče. Databáze funguje primárně v operační paměti počítače, takže funguje velmi rychle a je velmi efektivní. Největší využití je pro ukládání do cache aby byl zrychlen chod aplikace. Databáze také podporuje poslouchání na určitých hodnotách a v reálném čase posílání hodnot na poslouchající klienty. Tato funkcionalita je nejideálnější s technologií WebSocket, nebo ideálněji s GraphQL komponentou "Subscription". 

\subsection{Docker}

\subsection{Kubernetes}
Kubernetes, also known as K8s, is an open-source system for automating deployment, scaling, and management of containerized applications. \cite{Kubernetes}

\subsection{TensorFlow}