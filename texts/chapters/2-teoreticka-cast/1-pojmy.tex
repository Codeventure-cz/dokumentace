\section{Pojmy}

\subsection{WebSocket}
WebSocket je komunikační protokol pro komunikaci mezi serverem a klientem. Výhodou oproti HTTP požadavku je to, že komunikace je na začátku navázána a pak se dají požadavky posílat obousměrně bez jakékoliv prodlevy.

\subsection{Frontend}
Frontend, neboli přední strana, je webové rozhraní které vidí koncový uživatel. Veškerý kód který se zde odehrává běží u klienta, nikoliv na severu. Uživatel je schopný se podívat na kód, který u něj běží a modifikovat jej.

\subsection{Backend}
Backend, neboli zadní strana, je kód, který běží na serveru. Na tuto část se pomocí API dotazuje frontend. Uživatel není schopný se podívat na kód a nemůže měnit jeho funkcionalitu.

\subsection{Kompilace}
Kompilace je proces, při kterém se člověkem čitelný kód převádí na počítačem čitelný kód, člověkem nikoliv. Většinou není možné získat kód zpět do člověkem čitelné podoby po tom co je zkompilován.

\subsection{Framework}
Framework je platforma, která poskytuje základ pro vývoj softwareových aplikací. \cite{Framework} Jedná se o šablonu programu, kterou pak programátor rozšiřuje vlastním kódem.

\subsection{Komponenta}
Komponenta je menší část aplikace, která poskytuje určitou funkcionalitu.

\subsection{Herní engine}
Herní engine je framework pro výtváření počítačových her.

\subsection{VPS}
VPS (Virtual Private Server) neboli virtuální soukromý počítač je označení pro virtuální počítač, který je prodáván jako služba.

\subsection{API}
API (Application programming interface) je rozhraní, které umožňuje předávání dat mezi frontendem a backendem.

\subsection{Škálování}
Škálovatelnost softwaru je měřítko toho, jak jednoduše lze část softwaru zmenšovat nebo zvětšovat. \cite{Skalovani} Například když máme moc velké využití backendu tak jsme schopní spustit tento backend na více serverech zároveň.

\subsection{JSON}
JSON (JavaScript Object Notation) je datový formát pro výměnu dat. Je jednoduše čitelný a je to jedna z možností jak přes API rozhraní posílat data. \cite{JSON}

\subsection{Transpilace}
Transpilace je proces, při kterém se vezme zdrojový kód napsaný v jednom programovacím jazyce a je převeden na podobný kód v jiném programovacím jazyce. \cite{Transpilace}

\subsection{Cache}
Cache, neboli mezipaměť, je úložné místo pro ukládání dočasných dat díky kterým se zrychlují weby, prohlížeče a aplikace. \cite{Cache} Díky tomu se po určitý čas nemusí provádět složité výpočetní operace.

\subsection{Open Source}
Open source software je software se zdrojovým kódem, který si může kdokoliv stáhnout, upravit nebo vylepšit. \cite{OpenSource}

\subsection{Databáze}
Databáze je místo, kam si aplikace ukládá data. Dále si dokáže data zpracovavát a následovně vracet zpět do aplikace.

\subsection{ORM}
ORM neboli Objektově relační mapování vytváří vrstvu mezi programovacím jazykem a databází. Pomocí něj si navrhneme jak by měla databáze vypadat a on pak řeší všechny dotazy na databázi. Výhodou tedy je, že nemusíme psát dotazy přímo na databázi, ale v rovnou v programovacím jazyku můžeme se stejnou sytaxí zavolat na databázi.

\subsection{CDN}
CDN neboli content delivery network (síť pro doručování obsahu) je síť počítačů rozmístěných po celém světě, které spolupracují na poskytování rychlého doručování internetového obsahu.\cite{CDN}
