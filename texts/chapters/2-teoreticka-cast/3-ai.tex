\section{Umělá inteligence}
Umělá inteligence, v angličtině Artificial Intelligence (zkráceně AI), je věda a inženýrství vytváření inteligentních strojů, zejména počítačových programů.\cite{What-is-AI} Toto téma je velmi rozsáhlé, a proto jej dělíme do těchto oblastí\cite{IBM-AI}:
\begin{itemize}
	\item Rozpoznání řeči
	\item Zákaznické služby
	\item Počítačové vidění
	\item Doporučovací algoritmy
	\item Automatické obchodování s akciemi
\end{itemize}

\subsection{Počítačové vidění}
Počítačové vidění umožňuje počítači získávání smysluplných informací z vizuálních vstupů (obrázky, videa). \cite{IBM-CV} Funguje velmi podobně jako lidské vidění - musí se učit, v tom mají, ale lidé výhodu celoživotního kontextu a tím pádem dokáží určovat více věcí zároveň.\par

\subsubsection{Strojové učení}
K učení je potřeba velké množství dat - obrázky nebo videa, na kterých jsou objekty, které má specifiká kolekce rozlišovat. Od trénující osoby je očekáván vstup s informacemi co za objekty je na obrázku, popřípadě i jejich souřadnice. Počítač následně odhaduje pomocí těchto vstupních dat obsah na obrázku. Takovému výstupu říkáme predikce.\par
Samotný proces učení probíhá tak, že počítač prochází pixely v oblastech a zdokonaluje své rozeznávání do doby, než jsou predikce dostatečně blízko vstupním datům. Tento proces může trvat několika minut i několik měsíců. Záleží na velikosti vstupních dat a výkonu počítače. Pro tyto výpočty je nejefektivnější využít grafickou kartu.\par
Když jsou predikce dostatečně blízko vstupním datům je proces učení ukončen a je z něj vygenerován model, který je dále použit pro predikci, ale není nadále zdokonalován. 

\subsection{Frameworky}
Vzhledem k velkému počtu oblastí umělé inteligence existuje nezpočet možných framewoků implementujících toto téma. Mezi nejznámější patří:
\begin{itemize}
	\item TensorFlow
	\item Caffe
	\item Torch
	\item Microsoft Cognitive Toolkit
\end{itemize}

Velké společnosti také implementují některé z výše uvedených frameworků do svých cloudových služeb. Například služba Microsoft Custom Vision podporuje export vytrénovaných modelů do formátu pro TensorFlow.