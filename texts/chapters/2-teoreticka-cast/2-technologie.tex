\section{Technologie}

\subsection{Unity}
Unity je herní engine určený hlavně pro nováčky, ale i pro pokročilé programátory. Jako programovací jazyk Unity využívá C\# vyvýjený firmou Microsoft. Tento programovací jazyk je jednoduchý pro nováčky a je jedním z nejzsnažších programovacích jazyků v současnosti na naučení. \cite{Csharp} Dají se v něm vytvářet 2D i 3D aplikace. Výsledná hra je spustitelná na mnoha platformách, od počítačů až po mobilní zařízení.

\subsection{JavaScript}
JavaScript je moderní programovací jazyk využívající se jak na frontendu, ale i na backendu. Jazyk je tzv. interpretovaný jazyk, což znamená, že kód je spustitelný rovnou ze zdrojového kódu a není potřeba jej kompilovat.

\subsection{TypeScript}
TypeScript je nadstavbou JavaScriptu, která poskytuje typování kódu. Díky tomu je ve výsledné aplikaci méně chyb spojených s typy, protože jsou kontrolovány během vývoje a sestavování aplikace. Ve výsledku je kód transpilován zpět do JavaScriptu.
TODO domyslet

\subsection{React.js}
React.js je JavaScriptový framework pro vývoj uživatelské rozhraní neboli UI. Tvoří se s ním dynamické webové aplikace nebo mobilní aplikace. Vývojářem tohoto frameworku je firma Meta (dříve známá jako firma Facebook). UI je děleno na menší komponenty, které mají svoji funkcionalitu a dají se mezi sebou jednoduše zaměňovat. Díky tomu není například potřeba načíst celou stránku při přechodu na jinou stránku, ale stačí pouze zaměnit komponenty které se na stránce změnily (např hlavička a patička zůstávají stále načteny, ale obsah stránky se změní).

\subsection{Next.js}
Next.js je JavaScriptový framework stavící na frameworku React.js. Je vytvářen společností Vercel. Tento framework usnadňuje vývojáři práci se stránkami a přidává nezpočet výhod nad použitím samothného Reactu.js. Hlavním lákadlem je takzvaný "Server Side Rendering" což znamená, že počáteční kód který, je poslán klientovi, je nejdříve vykreslen na serveru, pak je odeslán v neživé podobě klientovi a až po té, co je stránka načtena tak se z něj stává dynamická stránka pomocí React.js. Mezi další výhody patří například automatická optimalizace velikosti obrázků podle toho, jak velké je klient potřebuje, aby se nemusely posílat zbytečně velké a neoptimalizované obrázky.

\subsection{Node.js}
Node.js je backendové prostředí běžící na JavaScriptovém kódu. Výhodou oproti starším programovacím jazykům které běží na serveru např. PHP je, že aplikace neustále běží na pozadí. To umožňuje o dost jednoduší práci s časem (např. načasování určitých událostí neboli Cronjobů) nebo práci v reálném čase pomocí WebSocketů.

\subsection{GraphQL}
GraphQL je je jazyk pro API rozhraní sloužící pro jednoduchou komunikací mezi frontendem/aplikací a backendem. Typy dotazů se dělí na 3 skupiny: Query: získání dat ze serveru, Mutation: změna dat na serveru a Subscription: zasílání dat v reálném čase pomocí WebSocketů.

\subsection{MongoDB}
MongoDB je dokumentová databáze, která je používaná na velké objemy dat. Data jsou zde ukládána do dokumentů ve formátu JSON. Databáze je jednoduše škálovatelná mezi více zařízeními v tak zvaném "Replica Setu" (data jsou kopírovaná mezi všemi zařízením, aby bylo možné při výpadku některého zařízení pokračovat dál v chodu aplikace). Replica set umožňuje i další pokročilé funkce.

\subsection{Redis}
Redis je databáze fungující na principu klíč a hodnota. Pod nějakým klíčovým slovem si můžeme uložit hodnotu a následně si můžeme tuto hodnotu zobrazit po zadání tohoto klíče. Databáze funguje primárně v operační paměti počítače, takže funguje velmi rychle a je velmi efektivní. Největší využití je pro ukládání do cache aby byl zrychlen chod aplikace. Databáze také podporuje poslouchání na určitých hodnotách a v reálném čase posílání hodnot na poslouchající klienty. Tato funkcionalita je nejideálnější s technologií WebSocket, nebo ideálněji s GraphQL komponentou "Subscription".

\subsection{Flutter}
Flutter je framework od Googlu pro vytváření krásných, nativních, multiplatformních aplikací z jednoho kódu. \cite{Flutter} Jako programovací jazyk je zde Dart, který vyvýjí také Google. Flutter cílí primárně na zařízení s operačním systémem Android nebo iOS, ale i na Windows, Linux, MacOS nebo web. Výhodou je možnost vykreslení každého pixelu na obrazovce jak chceme.

\subsection{Kontejner}
Kontejner je standardizovaná část softwaru ve které je zabalený kód a všechny jeho potřebné závislosti, aby bylo možné aplikaci rychle spuštět a aby byla aplikace konzistentní mezi různými prostředími. \cite{Kontejner}

\subsection{Docker}
Docker je platforma pro vytváření, správu a běh kontejnerů. Pomocí takzvaných "Dockerfile" je možné definovat, jak bude výsledný kontejner sestaven a jak bude fungovat. Lze pomocí něj také spouštět samotné kontejnery ať už námi sestavené tak sestavené komunitou. Díky němu lze spustit v podstatě jakoukoliv aplikaci pomocí jednoho příkazu.

\subsection{Kubernetes}
Kubernetes je platforma pro automatickou správu kontejnerů, jejich škálovatelnosti nebo automatického nasazení při nové verzi. Je přizpůsobena pro běh produkčních aplikací, tím pádem je velmi robustní. Je schopen přizpůsobit počet kontejnerů podle aktuální zátěže. Původním výrobcem byl Google kvůli jejich platformě GCP (Goole Cloud Platform), ale nyní patří pod Cloud Native Computing Foundation. Díky tomu, že je Kubernetes open-source, vznikají podobné verze, které se trochu liší od původní verze. Příkladem jsou například: K3s, Minikube, OpenShift, Mirantis, Amazon EKS a další.

\subsection{TensorFlow}
TensorFlow je platforma pro strojové učení. Slouží pro trénování umělé inteligence a vytváření neuronových sítí. Tyto neuronové sítě jsou pak při přiložení vstupních dat využité k predikci. Pro TensorFlow se píše kód primárně v programovacím jazyce Python, ale má i další implementace, například: JavaScript (TensorFlow.js), nebo Java (TensorFlow Lite).

\subsection{Blender}
Blender je open source software pro tvorbu počítačové grafiky. Lze využít na 3D modelování, 2D/3D animace a mnoho dalšího.

\subsection{Figma}
Figma slouží v projektu jako návrhové řešení a pro tvorbu designů mobilní aplikace, či webové. Díky ní lze vytvořit funkční prototyp pro prvotní testování a přehlednost UI se zlepšením UX.

\subsection{Notion}
V aplikaci Notion se dá přehledně spravovat projektové řešení. Má spostu výhod, jako například: sdílené listy s úkoly, které je potřeba vypracovat. Notion udržuje přehled a celkovou čistotu pro efektivnější komunikaci.

\subsection{Photoshop}
Aplikace Photoshop patří do balíčku od společnosti Adobe. Tento software nabízí mnoho možností při tvoření grafických podkladů.

\subsection{Illustrator}
Illustrator je software také os společnosti Adobe pro tvorbu počítačové grafiky. Jeho největší předností je tvorby vektorové grafiky.
