Hra Codeventure má za cíl zaujmout uživatele formou hry, představit svět programování a logického myšlení. Hráč se ve hře zapojuje do plnění úkolů, kterými prochází jednotlivými sekvencemi hry, získává odměny, které ho motivují ke splnění cílového úkolu.\par Hra je vystavěna jako propojení mobilního zařízení, vetší zobrazovací plochy a kartiček s povely, které hráč využívá k průchodu hrou. Zapojením příkazových kartiček do hry využívá hráč i své kreativity k dosažení určeného cíle, protože ve hře můžeme nalézt úrovně, které mají otevřené řešení. Složitost úrovní se stupňuje s postupem hrou.\par Při přenosu příkazů z kartiček do hry se používá umělá inteligence, která rozpoznává jednotlivé příkazy a ty následně přenáší hráči na cílové zařízení. Hráč poté vidí vyhodnocení zvoleného postupu. \par
V momentu správného sestavení příkazových kartiček hra otevírá hráčí další kolo a odmění ho.\par Pokud se jedná o nesprávnou sekvenci, tak hra hráče upozorní na chybné řešení a průchod hrou se pak musí opakovat za použití jiné kombinace kartiček.\par Na konci správně vyřešené úrovně je hráč hodnocen i za nejefektivnější způsob průchodu hrou, aby měl motivaci se dále zlepšovat a rozvíjet své vnímání logických příkazů směřujících k co nejlepšímu výsledku.