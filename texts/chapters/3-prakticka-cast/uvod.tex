Hra Codeventure má za cíl zaujmout uživatele formou hry a představit mu svět programování a logického myšlení. Hráč se zapojuje do plnění úkolů, kterými prochází v jednotlivých sekvencích hry, a postupně získává odměny, které ho motivují ke splnění cílového úkolu.\par
Hra je vystavěna jako propojení mobilního zařízení, vetší zobrazovací plochy a kartiček s povely, které hráč využívá k průchodu hrou. Zapojením příkazových kartiček využívá hráč i své kreativity k dosažení určeného cíle, protože ve hře můžeme nalézt úrovně, které mají otevřené řešení. Složitost úrovní se postupně stupňuje.\par
Při přenosu příkazů z kartiček do hry se používá umělá inteligence, která rozpoznává jednotlivé příkazy, a ty následně přenáší hráči na cílové zařízení. Hráč poté vidí vyhodnocení zvoleného postupu.\par
V momentě správného sestavení příkazových kartiček hra otevírá hráči další kolo a odmění ho.\par
Pokud se jedná o nesprávnou sekvenci, hra hráče upozorní na chybné řešení a pokus se musí opakovat za použití jiné kombinace kartiček.\par
Na konci správně vyřešené úrovně je hráč hodnocen i za nejefektivnější způsob průchodu hrou, aby měl motivaci se dále zlepšovat a rozvíjet své vnímání logických příkazů směřujících k co nejlepšímu výsledku.