Hra Codeventure je o tom, aby zaujala uživatele formou hry, představila svět programování a logického myšlení.\par
Hráč se v této hře zapojuje do plnění úkolů, kterými prochází jednotlivými sekvencemi hry, získává odměny, které ho motivují ke splnění cílového úkolu.\par Jedná se o propojení mobilního zařízení, vetší zobrazovací plochy a kartiček s povely, které hráč využívá k průchodu hrou. Zapojením příkazových kartiček do hry využívá hráč i své kreativity k dosažení určeného cíle, protože ve hře můžeme nalézt úrovně, které mají otevřené řešení. Složitost úrovní se stupňuje přímo úměrně s postupem hrou.\par Při přenosu příkazů z kartiček do hry se používá umělá inteligence, která rozpoznává jednotlivé příkazy a ty následně přenáší hráči na cílové zařízení a poté vidí vyhodnocení zvoleného postupu. Uvede postavičku do pohybu, která při správném sestavení sekvence kartiček dostane postavičku do cílového bodu, nebo vyhodnotí sestavu povelů, jako chybnou.\par
V momentu správného sestavení příkazových kartiček hra otevírá hráčí další kolo a odmění ho.\par Pokud se jedná o nesprávnou sekvenci, tak hra hráče upozorní na chybné řešení a průchod hrou musí opakovat za použití jiné kombinace kartiček.\par Na konci správně vyřešené úrovně je hráč hodnocen i za nejefektivnější způsob průchodu hrou, aby měl motivaci se dále zlepšovat a rozvíjet své vnímání logických příkazů směřujících k co nejlepšímu výsledku.