\section{Hra}
Hra je vytvořena pomocí herního enginu Unity. Aby bylo možné ji spustit v prohlížečí, je zde použa možnost exportu do webového prostředí neboli \uv{WebGL}. Většina funkcí, které hra obsahuje nejsou dělány pomocí grafického editoru, ale pomocí kódu z důvodu proměnlivosti zobrazovaných dat.

\subsection{Komunikace}
Klíčovou funkcí hry je komunikace mezi backendem a hrou aby bylo možné zobrazovat úkony hráče. Tuto práci výrazně zlehčuje v předchozí kapitole již zmíněná knihovna \uv{react-unity-webgl}\cite{react-unity-webgl}. Díky této knihovně není třeba komunikovat s backendem dalším způsobem a je možné použít již existující spojení mezi frontendem a backendem.\par
Komunikace pak může probíhat oběma směry. Směr hra na backend je použit pouze jednou, a to při načtení když hra potřebuje ostrovy. Této akce je docíleno pomocí připravené funkce v Reactu, která pomocí mutace \uv{getUnityIslands} kontaktuje server, aby zaslal zpět nové ostrovy. Na tu funkci se z prostředí Unity volá pomocí metody \uv{GetIslands}.\par
Druhý směr komunikace je častější a to backend na hru. Backend má možnost volat na jakoukoliv veřejnou metodu umístěnou v komunikačním skriptu. Jako argument může přiložit data, která chce předat. Tyto data jsou většinou ve formátu JSON a je třeba je deserializovat. Používá se například pro přejetí kamery na jiný ostrov, nebo nastavení objektů do scény se zadáním.

\subsection{Načítání modelů}
Aby byla hra co nejmenší a aby se nenačítala velmi dlouho je potřeba rozdělit části hry do více částí a následně tyto části načítat podle potřeby hráče. Pro tento účel je slouží balíčkový systém zvaný \uv{AssetBundle}\cite{AssetBundle}. V těchto balíčcích jsou obsažené modely a animace k jednotlivým ostrovům, které se načítají podle toho, co uživatel zvolí na kontroléru.

\subsection{Animace}
Poslední důležitou částí jsou animace. Animace jsou přidávány na modely díky pomocí koster. Díky tomu je možné použít jeden model s více animacemi. Tyto animace jsou pak spouštěny pomocí koprogramů, které na sebe navzájem volají. 
