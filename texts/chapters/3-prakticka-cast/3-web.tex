\section{Web}
Web neboli frontend je hlavní částí, kterou uživatel vidí. Jeho úkoly jsou: zobrazit informace, správa uživatelského účtu a zobrazování hry. Klíčové prvky jsou rozepsané v následujících podkapitolách.

\subsection{Statická část}
Web je postaven na frameworku Next.js, který používá pro zobrazování React.js. Cílem bylo vytvořit web co nejrychlejší a nejspolehlivější.\par
Aby byl web co nejrychlejší je potřeba, aby všechny části byly předem vygenerované a aby nebylo nuté zbytečně čekat při dotazu. Next.js v základu stavu počítá s tím, že obrázky jsou optimalizované při času jejich dotazu. Tato funkce je užitečná na stránkách s proměnlivým obsahem, nikoliv na staticky generovaných stránkách. Proto jsme aplikovali trochu netradiční řešení, a to optimalizace obrázků při generování webu. Díky tomu není potřeba žádný speciální server a je potřeba jen všechny soubory někam umístit.\par
Pro co nejrychlejší a nespolehlivější běh webu je potřeba jej umístit na nějaký server, který bude schopen rychle odpovídat velkému množství lidí. Pro tento účel byl vybrán hosting Cloudflare Pages\cite{Cloudflare-pages} s jejich vlastní CDN\cite{Cloudflare-cdn}. 

\subsection{Správá účtu}
Důležitou částí webu je přihlašovací systém a následná správa účtu. Pro přihlášení je možné použít email a heslo (pomocí GraphQL mutace \uv{login}) nebo QR kód, který se dá naskenovat pomocí mobilní aplikace (subscription \uv{qrLogin} a stejnojmenná mutace). Dále je zde možnost zaregistrovat se (mutace \uv{register}) a resetovat heslo pomocí zaslání emailu (mutace \uv{sendResetPassword}).\par
Když je uživatel přihlášen, je možné odhlásit se (mutace \uv{logout}) nebo aktuální účet smazat (mutace \uv{deleteAccount}).

\subsection{Hra}
Jelikož je hra vytvořena pomocí Unity je potřeba ji zaintegrovat do webového rozhraní tak, aby si uživatel nevšimnul, že přešel do jiného prostředí. Pro tento účel slouží knihovna \uv{react-unity-webgl}\cite{react-unity-webgl}, která poskytuje integraci kódu v Unity s Reactím kódem. Komunikaci mezi backendem a hrou obsluhuje subscription \uv{unityCommunication}\par
Informace o vnitřním fungování hry jsou rozepsané v další pod kapitole.
