Zabezpečení je důležitou částí každé webové aplikace. Kvůli rozlehlosti tohoto projektu je potřeba, aby byla každá část zabezpečená ve všech ohledech.\par
Nejvíce zranitelná část je část webová. Nejvýznamnější bezpečností rizika zaznamenává nadace OWASP\cite{OWASP} s jejich seznamem \uv{Top 10 Web Application Security Risks}\cite{OWASP-top-ten}. V seznamu je 10 nejčastějších bezpečnostích rizik, ktereeá byla v posledních letech nejfrekventovanější.\par
Backendová i frontendová část má většinu těchto chyb odlazených a neměly by být zneužitelné. Zde jsou konkrétní příklady některých bodů:
\begin{itemize}
	\item \textbf{Kryptografické selhání}\cite{CryptographicFailures}: Všechny části předpokládají komunikaci pomocí šifrovaného protokolu (HTTPS). Certifikáty jsou podepsané od kryptografických autorit - Let's Encrypt a Cloudflare. Hesla jsou v databázi hashovaná pomocí šifrovací funkce \uv{bcrypt}\cite{bcrypt}.
	\item \textbf{Injection}\cite{Injection}: Rozhraní GraphQL validuje integritu datových typů, při práci s textem je tento text zkontrolován pomocí knihovny \uv{validator}\cite{Validator} pro správný formát. ORM framework Prisma vytváří bezpečné dotazy, aby nebylo možné vytvořit NoSQL inject.
	\item \textbf{Zranitelné a zastaralé komponenty}\cite{VulnerableAndOutdatedComponents}: Všechny balíčky, které jsou použity, jsou z oficiálních zdrojů (NPM, pub.dev). Balíček musí splňovat alespoň jednu z těchto specifikací: je na dané platformě populární (vysoký počet stažení), má bezproblémový zdrojový kód (musí být manuálně prověřeno) nebo vlastní balíček. Bezproblémovost balíčků automaticky kontroluje GitHub Dependabot, který monitoruje použité balíčky a při výskytu problému vytváří automaticky opravy.
\end{itemize}
