Pro hraní hry je potřeba mobilní telefon s aplikací Codeventure, herní kartičky a zobrazovací zařízení, na kterém lze spustit webová aplikace. Nejlepší herní požitek hráč získá při velké ploše zobrazovacího zařízení. Prvním krokem je vytvoření herního účtu, kde se ukládá jednotlivý postup hráče hrou. Následuje stažení mobilní aplikace, kde se hráč přihlásí již vytvořeným účtem. Poté se přesune ve webové aplikaci do hry, kde uvidí první ostrov. Dále hráč využívá pouze mobilní telefon, a to jako ovladač. Na mobilím zařízení si zvolí ostrov a úroveň, kterou chce splnit. Před spuštěním levelu se hráči ukáže slovní zadání úkolu. Dalším krokem je vyobrazení herního zadání ve webové aplikaci a v mobilní spuštění fotoaparátu pro naskenování kartiček. Hráč musí pro úspěšné splnění úrovně seskládat validní a správnou sekvenci kartiček a poté ji naskenovat. Na obr. \ref{fig:pohled-hrace} můžeme vidět herní zadání a naskenovanou karetní sekvenci. Po vyhodnocení se hráči na mobilním telefonu zobrazí, jestli uspěl, či nikoliv. Jednotlivé úrovně může opakovat, nebo může zvolit jinou úroveň. Hra se snaží hráče nabádat k používání sofistikovanějších řešení a díky tomu jej posouvat. 

\begin{figure}[h]
    \centering
    \includegraphics[width=0.6\textwidth]{img/pohled-hrace.png}
    \caption{Pohled hráče.}
    \label{fig:pohled-hrace}
\end{figure}