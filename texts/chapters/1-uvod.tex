\chapter{Úvod}

Celý svět kolem nás používá počítačové technologie s naprogramovanými aplikacemi. Právě kvůli tomu se s programováním setkáváme každý den, aniž bychom o tom věděli. V současné době se převádí mnoho školních činností do elektronické formy. Učebny jsou vybavovány interaktivními tabulemi, žáci pracují na počítačích či tabletech. Přesto je ale manipulace s reálnými objekty velmi důležitá pro rozvoj dětí nejen v předškolním, ale i mladším školním věku. Marie Montessori ve své koncepci vzdělávání dokonce tvrdí, že právě ruce napomáhají rozvoji intelektu. Když dítě používá své ruce, získává množství podnětů a zážitků a rozvíjí tak své vědomí a intelekt.\cite{Montessori}
\par
Dle vývojových psychologů\cite{VyvojovaPsychologie} již ve druhé třídě začíná být logika u většiny dětí obvyklým nástrojem k postihování reality. Přesto je dostatečně účinná pouze při úvahách o něčem konkrétním a názorném. A proto považují psychologové za nutné doplňovat výuku názornou demonstrací, nejlépe i s určitým podílem dětské aktivity. Dále se v \cite{ASE} uvádí, že z propojení neurologických a pedagogických výzkumů vyplývá, že kvalitní učení nastává v případě, že je zapojen doslova celý meze, nejedná se tedy o aktivity zaměřené jen na levou či pravou hemisféru. Vhodné je zapojit u žáků všechny smysly, a to nikoli proto, aby se zvýšila zábavnost či zajímavost dané činnosti, ale protože se tím sníží kognitivní náročnost, zejména na pracovní paměť.
\par
Na základě výše zmíněných faktů jsme se rozhodli vyvynout edukativní hru, protože jedním z nejlepších a nejpřirozenějších způsobů, jak se dítě naučí jakékoli dovednosti, je hra, a není důvod, aby toto pravidlo neplatilo i pro algoritmické myšlení a programování. V dnešní době existují hry jako LightBot, Kodable a mnoho dalších, které děti rozvíjejí právě ve zmíněných okruzích. Často se však jedná o hry, které jsou určeny pro menší zařízení, jako tablety, či telefony a dítě prochází hrou pouze za pomocí dotykové obrazovky.
\par
Tzv. gamifikace učení je celosvětovým trendem s pozitivními výsledky, proto jsme se stejnou cestou vydali i my. Cílem práce je vytvořit hru, která rozvine logické myšlení a základy programování. Celkový návrh hry je navržený tak, abychom dokázali děti co nejefektivněji naučit zapojit logické myšlení a rozvinuli jejich dovednost základů programování.