\chapter{Úvod}

Celý svět kolem nás používá počítačové technologie s naprogramovanými aplikacemi. Právě kvůli tomu se s programováním setkáváme každý den, aniž bychom o tom věděli. Jedná se opravdu o užitečnou schopnost, kterou chceme rozvíjet.\par
Věříme, že je nejlepší začít v raném věku a to u dětí ve školním období. Je rozumné, když se malé děti setkají s programováním co nejdříve. Nejedná se pouze o schopnost umět naprogramovat aplikaci, ale především zapojení a trénování logického myšlení.\par
Jedním z nejlepších a nejpřirozenějších způsobů, jak se dítě naučí jakékoli dovednosti, je hra a není důvod, aby toto pravidlo neplatilo i pro algoritmické myšlení a programování. V dnešní době existují hry, jako LightBot, Kodable a mnoho dalších, které děti rozvíjejí právě ve zmíněných okruzích, často se však jedná o hry, které jsou vyvíjeny pro menší zařízení, jako tablety, či telefony a dítě prochází hrou pouze za pomocí dotykové obrazovky.\par
Dětem, především v mladším školním věku, obecně pomáhá, když můžou při učení se nové věci manipulovat s fyzickým předmětem, protože zapojují více smyslů a díky kontaktu a prožitku se učí nové věci více kvalitně a v kratším časovém intervalu.\par
Tzv. gamifikace učení je celosvětovým trendem s pozitivními výsledky, proto jsme se stejnou cestou vydali i my. Cílem práce je vytvořit hru, která rozvine logické myšlení a základy programování. Celkový návrh hry je navržený tak, abychom dokázali děti co nejefektivněji naučit zapojit logické myšlení a rozvinuli jejich schopnost základů programování.