\chapter{Úvod}

Celý svět kolem nás používá počítačové technologie s naprogramovanýmy programy. Právě kvůli tomu se s programováním setkáváme každý den, aniž bychom o tom věděli. Jedná se opravdu o užitečnou schopnost, kterou chceme rozvíjet. Samozřejmě je nejlepší začít v raném věku a to u dětí ve školním období. Je rozumné, když se malé děti setkají s programováním co nejdříve. Nejedná se pouze o schopnost umět naprogramovat aplikaci, ale především zapojení a trénování logického myšlení. Nejlepším způsobem jak se těmto schopnostem naučit je jednoznačně formou hry. V dnešní době existují hry, jako LightBot, Kodable a mnoho dalších, které děti rozvíjejí právě ve zmíněných okruzích. Často se jedná o hry, které jsou na menší zařízení, jako tablety, či telefony a dítě prochází hrou pouze za pomocí dotykové obrazovky. Dětem obecně pomáhá, když můžou při učení se nové věci manipulovat s fyzickým předmětem, protože zapojují více smyslů a díky kontaktu a prožitku se učí nové věci více kvalitně a v kratším časovém intervalu. My jsme chtěli do této problematiky také přispět, a proto jsme vymysleli a poté vytvořili hru s názvem Codeventure. Celkový návrh hry je navržený tak, abychom dokázali děti co nejefektivněji naučit zapojit logické myšlení a rozvinuli jejich schopnost základů programování. Hráč pro průchod hrou využívá fyzické kartičky, které představují jednotlivé herní kroky kvůli lepšímu učení. Pro nejlepší herní prožitek je vhodné si hru zapnout na velké zobrazovací ploše a využívat mobilní aplikaci pouze jako ovladač, který slouží pro přepínání a plnění jednotlivých herních úrovní. Hra využívá také umělou inteligenci, která rozpoznává jednotlivé kartičky naskenované hráčem po složení herní sekvence. Díky hře Codeventure dítě zažije nezapomenutelný zážitek, při kterém se naučí využívat logické myšlení, základy programování a ještě se přitom zabavý v kouzelném světě Codeventure, který čeká opravdu na kohokoliv.