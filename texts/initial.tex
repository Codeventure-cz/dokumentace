\begin{titlepage}
    \bfseries{
        \begin{center}
            \LARGE{STŘEDOŠKOLSKÁ ODBORNÁ ČINNOST}

            \vspace{14pt}
            \large{
                Obor č. 18: Informatika
            }

            \vspace{0.35 \textheight}

            \LARGE{
			Codeventure
            }

            \vspace{0.35\textheight}
        \end{center}
        
        \noindent\Large{Michal Maděra, Josef Kahoun}

        \noindent\Large{Pardubický kraj \hspace{\stretch{1}}  Pardubice, 2022}
        
            
    }
\end{titlepage}

\cleardoublepage

%% Úvodní stránka s informacemi
{\bfseries
    \begin{center}
        \LARGE{STŘEDOŠKOLSKÁ ODBORNÁ ČINNOST}

        \vspace{14pt}
        {\large
            Obor č. 18: Informatika
        }

        \vspace{0.3 \textheight}

        \LARGE{
        Codeventure
        }

        \LARGE{
        Codeventure
        }

        \vspace{0.24\textheight}
    \end{center}  
}
{\Large
    \noindent\textbf{Autoři:} Michal Maděra a Josef Kahoun\\
    \textbf{Škola:} DELTA - Střední škola informatiky a ekonomie, s.r.o.\\
    \textbf{Kraj:} Pardubický kraj\\
    \textbf{Konzultanti:} RNDr. Jan Koupil, Ph.D. a akademický malíř Daniel Václavík\\
}

\noindent Pardubice, 2022

\cleardoublepage

\noindent{\Large{\bfseries{Prohlášení}}}

\noindent Prohlašujeme, že jsme svou práci SOČ vypracovali samostatně a použili jsme pouze prameny a literaturu uvedené v seznamu bibliografických záznamů.

\noindent Prohlašujeme, že tištěná verze a elektronická verze soutěžní práce SOČ jsou shodné. 

\noindent Nemáme závažný důvod proti zpřístupňování této práce v souladu se zákonem č. 121/2000 Sb., o právu autorském, o právech souvisejících s právem autorským a o změně některých zákonů (autorský zákon) ve znění pozdějších předpisů. 

\vspace{24 pt}

\noindent V Pardubicích dne x. xxxx 2022 \dotfill{}\hspace{\stretch{0.5}} 

\hspace{6cm} Michal Maděra, Josef Kahoun

\cleardoublepage

\vspace*{0.8\textheight}
\noindent{\Large{\bfseries{Poděkování}}}

\noindent
Chtěli bychom poděkovat našim učitelům: RNDr. Janu Koupilovi, Ph.D. za vedení projektu po technické stránce a akademickému malíři Danielu Václavíkovi za vedení práce po grafické stránce. Dále bychom chtěli poděkovat všem lidem, kteří se zapojili testovací fáze našeho projektu.

\cleardoublepage

\noindent{\Large{\bfseries{Abstrakt}}}

\noindent Práce SOČ dokumentuje návrh a vývoj hry Codeventure zaměřené na rozvoj algoritmických a programovacích dovedností u dětí mladšího školního věku. Realizace projektu zahrnuje backendovou část, mobilní aplikaci v roli ovladače hry a webový frontend v roli displeje. Součástí řešení je využití umělé inteligence pro strojové rozpoznání obrazu.

\vspace{18pt}

\noindent{\Large{\bfseries{Klíčová slova}}}

\noindent Programování; GraphQL; TensorFlow 

\vspace{18pt}

\noindent{\Large{\bfseries{Abstract}}}

\noindent The main goal of this project is to create an educational game for primary school students, in which they learn how to logically think about code.

\vspace{18pt}

\noindent{\Large{\bfseries{Keywords}}}

\noindent Programming; GraphQL; TensorFlow

\cleardoublepage

\tableofcontents